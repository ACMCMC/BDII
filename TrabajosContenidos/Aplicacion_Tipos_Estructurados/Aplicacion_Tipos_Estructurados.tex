\documentclass[12pt, a4paper]{article}

\usepackage[spanish]{babel}
\usepackage{listings}
\usepackage[utf8]{inputenc}
\usepackage{titling}
\usepackage{enumitem}
\usepackage{fancyhdr}
\usepackage{xcolor}
\usepackage{geometry}
\usepackage{graphicx}
\usepackage{hyperref}
\usepackage{cite}
\usepackage{url}
\usepackage{fix-cm}
\usepackage{tikz}

\usepackage{etoolbox}
\patchcmd{\thebibliography}{\section*{\refname}}{}{}{}

%%search -> (?:url=\{)(.*)(:?\})
%%replace -> howpublished={\\url{$1}}
\geometry{a4paper, left=4em, right=4em, top=0em, bottom=4em}

\lstset{
    frame=single,
    breaklines=true,
    numbers=left,
    keywordstyle=\color{blue},
    numbersep=15pt,
    numberstyle=,
    basicstyle=\linespread{1.0}\selectfont\ttfamily,
    commentstyle=\color{gray},
    stringstyle=\color{orange},
    identifierstyle=\color{green!40!black},
}

\setlength{\parindent}{4em}
%%\setlength{\parindent}{0em}
\setlength{\parskip}{0.8em}
    
%%\renewcommand{\familydefault}{phv} %%Seleccionamos Helvetica
    
\lstdefinestyle{console}
{
    numbers=left,
    backgroundcolor=\color{violet},
    %%belowcaptionskip=1\baselineskip,
    breaklines=true,
    %%xleftmargin=\parindent,
    %%showstringspaces=false,
    basicstyle=\footnotesize\ttfamily,
    %%keywordstyle=\bfseries\color{green!40!black},
    %%commentstyle=\itshape\color{green},
    %%identifierstyle=\color{blue},
    %%stringstyle=\color{orange},
    basicstyle=\scriptsize\color{white}\ttfamily,
}

\newcommand{\SQLObligatorio}[1]{\textcolor{blue}{\textbf{#1}}}

\newcommand{\size}[2]{{\fontsize{#1}{0}\selectfont#2}}
    
\title{\size{15pt}{Aplicación de tipos de datos estucturados en SQL} \vspace{-2ex}}
\date{\vspace{-5ex}}
%\author{\size{15pt}{Aldán Creo Mariño} \vspace{-5ex}}

\bibliographystyle{plain}
    
\begin{document}

\maketitle
\thispagestyle{empty}

\vspace{-10ex}

En este documento, veremos un ejemplo de \textbf{cómo crear un tipo de datos estucturado en SQL}, a partir de su estándar de 1999. Vamos a partir del ejemplo de la clase \texttt{Usuario} del primer proyecto de la asignatura (la biblioteca), que en Java se definía como (he modificado el ejemplo para que sea más ilustrativo en la explicación):

\begin{center}
    \begin{minipage}[c]{0,7\textwidth}
\begin{lstlisting}[language=Java]
class Usuario {
    private String idUsuario;
    private String clave;
    private String nombre;
    private String direccion;
    private String email;
    private TipoUsuario tipo;
    public String getClaveEncriptada() {
        return Encriptador.encriptar(clave);
    } }
\end{lstlisting}
\end{minipage}
\end{center}

En SQL, podemos crear un tipo de datos estructurado similar usando la siguiente sintaxis:

\begin{center}
    \begin{minipage}[c]{0,7\textwidth}
\begin{lstlisting}[language=SQL]
CREATE TYPE Usuario AS (idUsuario text, clave text, nombre text, direccion text, email text, tipo TipoUsuario) FINAL
METHOD getClaveEncriptada() RETURNS text;
CREATE INSTANCE METHOD getClaveEncriptada() RETURNS text FOR Usuario BEGIN
    RETURN crypt(self.clave, gen_salt('algoritmo')); END;
\end{lstlisting}
    \end{minipage}
\end{center}

De esta forma, la creación de una tabla de usuarios se reduciría a:

\begin{center}
    \begin{minipage}[c]{0,7\textwidth}
\begin{lstlisting}[language=SQL]
CREATE TABLE usuarios AS (u Usuario);
\end{lstlisting}
    \end{minipage}
\end{center}

Y una consulta del nombre y la clave encriptada del usuario con un \texttt{idUsuario} concreto:

\begin{center}
    \begin{minipage}[c]{0,7\textwidth}
\begin{lstlisting}[language=SQL]
SELECT u.nombre, getClaveEncriptada() FROM usuarios WHERE u.idUsuario='id';
\end{lstlisting}
    \end{minipage}
\end{center}

Como se puede apreciar, la correspondencia entre la representación en lenguaje SQL y en lenguaje Java de la clase \texttt{Usuario} es considerablemente alta. Esto permite a los programadores del paradigma orientado a objetos realizar diseños más naturales, y que se integran más próximamente con las aplicaciones cliente.

Existen más opciones para la sintaxis que no tengo espacio para cubrir en este documento breve, así que recomiendo recurrir a la bibliografía para complementar esta explicación.

\nocite{*}
\bibliography{export}

\end{document}