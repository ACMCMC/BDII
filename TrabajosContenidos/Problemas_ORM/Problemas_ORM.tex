\documentclass[12pt, a4paper]{article}

\usepackage[spanish]{babel}
\usepackage{listings}
\usepackage[utf8]{inputenc}
\usepackage{titling}
\usepackage{enumitem}
\usepackage{fancyhdr}
\usepackage{xcolor}
\usepackage{geometry}
\usepackage{graphicx}
\usepackage{hyperref}
\usepackage{cite}
\usepackage{url}
\usepackage{fix-cm}
\usepackage{tikz}

\usepackage{etoolbox}
\patchcmd{\thebibliography}{\section*{\refname}}{}{}{}

%%search -> (?:url=\{)(.*)(:?\})
%%replace -> howpublished={\\url{$1}}
\geometry{a4paper, left=4em, right=4em, top=0em, bottom=4em}

\lstset{
    frame=single,
    breaklines=true,
    numbers=left,
    keywordstyle=\color{blue},
    numbersep=15pt,
    numberstyle=,
    basicstyle=\linespread{1.0}\selectfont\ttfamily,
    commentstyle=\color{gray},
    stringstyle=\color{orange},
    identifierstyle=\color{green!40!black},
}

\setlength{\parindent}{4em}
%%\setlength{\parindent}{0em}
\setlength{\parskip}{0.8em}
    
%%\renewcommand{\familydefault}{phv} %%Seleccionamos Helvetica
    
\lstdefinestyle{console}
{
    numbers=left,
    backgroundcolor=\color{violet},
    %%belowcaptionskip=1\baselineskip,
    breaklines=true,
    %%xleftmargin=\parindent,
    %%showstringspaces=false,
    basicstyle=\footnotesize\ttfamily,
    %%keywordstyle=\bfseries\color{green!40!black},
    %%commentstyle=\itshape\color{green},
    %%identifierstyle=\color{blue},
    %%stringstyle=\color{orange},
    basicstyle=\scriptsize\color{white}\ttfamily,
}

\newcommand{\SQLObligatorio}[1]{\textcolor{blue}{\textbf{#1}}}

\newcommand{\size}[2]{{\fontsize{#1}{0}\selectfont#2}}
    
\title{\size{15pt}{Problemas de ORM} \vspace{-2ex}}
\date{\vspace{-5ex}}
%\author{\size{15pt}{Aldán Creo Mariño} \vspace{-5ex}}

\bibliographystyle{plain}
    
\begin{document}

\maketitle
\thispagestyle{empty}

\vspace{-10ex}

\textbf{ORM} (\textit{Object-Relational Mapping}) es un mecanismo que permite que el lenguaje cliente de la base de datos (tomemos Java como ejemplo) establezca una relación de mapeo entre atributos de las clases definidas en el lenguaje, y atributos guardados en una base de datos relacional. En este documento, no voy a centrarme en explicar en qué se basa o cómo funciona (existen otros documentos al respecto), sino que me voy a centrar en comentar sus debilidades e inconvenientes.

Uno de las críticas principales a ORM es que hace que el programador tienda a centrarse únicamente en la parte de programación de la aplicación cliente. Es decir, lo más adecuado suele ser intentar poner la mayor parte del esfuerzo del lado de la programación de la base de datos, porque, como hemos visto en el proyecto en grupo que hemos tenido que desarrollar en la asignatura, suele traducirse en una mayor fiabilidad del sistema de bases de datos. Es más, existen ORM que incluso se encargan de generar automáticamente el esquema de la base de datos. Esta clase de enfoques suelen dar lugar a modelos subóptimos, ya que el diseñador de la base de datos es el que, en última instancia, conoce adecuadamente la lógica de negocio, y por tanto, sabe cómo sería la forma correcta de modelar la base de datos.

Además, los ORM suelen ser subóptimos en cuanto a tiempos de acceso y actualización. Por ejemplo, una actualización en masa de 1000 elementos podría ser realizada, dependiendo de cómo se haya implementado el propio ORM, como 1000 transacciones separadas, que generarían una gran carga en el sistema. Igualmente, los ORM a veces solicitan demasiados datos de la base de datos. Por ejemplo, si se le solicita obtener una entidad, lo que hará será un \texttt{SELECT} de todos los datos referentes a dicha entidad, cuando en ocasiones podría bastarnos con un subconjunto.

Aparte, los ORM ayudan a olvidar la cuestión de las autorizaciones. Si el programador se enfoca en programar la aplicación cliente, es mucho más probable que no vea la necesidad de definir una política coherente de permisos en la base de datos. Esto podría ser un punto vulnerable del sistema, en caso de que un atacante obtuviera acceso al mismo.

Otro problema de los ORM es que introducen dependencias en la representación de los datos del lado del modelo relacional. Aunque normalmente se encargan de ``traducir'' los tipos de datos de las clases a los correspondientes en la base de datos, no siempre es posible o está permitido. Por ejemplo, un tipo de datos que suele ser problemático son las enumeraciones o las marcas temporales. Un atributo de tipo Date en Java podría corresponderse en SQL con \texttt{timestamp}, \texttt{timestamptz}, etcétera. Por tanto, debería ser el programador el que decida explícitamente la representación del tipo de datos. Algunos ORM permiten hacer estas definiciones manualmente, pero con la desventaja de romper la abstacción entre la aplicación cliente y la base de datos.

Finalmente, es igualmente importante el hecho de que ORM suele presentar problemas a la hora de escalar la base de datos. En sistemas muy complejos, suelen ser necesarios joins y consultas complejas para recuperar determinados conjuntos de datos. ORM suele fallar en este tipo de situaciones, recurriendo normalmente a realizar múltiples consultas simples y combinar sus resultados del lado del cliente. Esto es notablemente negativo para la eficiencia de la aplicación, ya que genera una sobrecarga en la base de datos.

En definitiva, el problema fundamental es que \textbf{intenta sustituír el trabajo en la base de datos}, y éste es \textbf{tremendamente necesario} para construir una aplicación eficiente y fiable. 

\nocite{*}
\bibliography{export}

\end{document}